\mytitle{To be able to not lose you valuable}
There was an incident on Wed 7 Nov and Thu 8 Nov that "Sim" forgot the place that she leaved her 'red bag from Kista' and a bunch of incidents that she couldn't find her keys, phone, chargers, food boxes, etc. I want to help her by finding her stuffs or noticing if she might be forget anything. I may be able to do a few not eveything because I also have my own stuffs to memorize. Moreover, I want her to take caremy stuffs in the future, such as "family money". How can she take care of our stuffs if she is very forgetful.

I've asked myself why I rarely has this kind of problems. There are a few philosophy that I used:
\begin{itemize}
\item {\bf Having stuffs to concern as few as possible} - the fewer the stuffs you have to concern, the smaller the chance to lost ones of them. Please imagine two guys who have different number of items in their bag and they have to use their items during their working hours. The first guy has only one item while the second guy has one hundred items. Who do you think has the higher risk to lose his item? It's obviously the second guy, isn't it? How can one person can pay attention to hundred items? Zero item is the ideal number because you don't have to concern about anything but it's not practical. So having as few stuffs as possible is the concept here. 
\item {\bf Making it routine and having a list of important items} - How can one will forget his important stuffs if he check them routinely? Please imagine a guy who check his stuffs before going to school, before leaving school. How can he lose his items? The question here might be "which one is importatnt?". Initially, the list may be small but it'll become bigger once you find other important stuffs. Please be aware that the bigger the list, the higher the effort you use to check your stuffs.
\item {\bf Searching for your items seqeuncially} - Please imagine a guy who can remember the moment he bought his stuff from a certain shop. He can remembered that he put down the bug whiel binding his shoelaces. He can remembered that he switched his bag from right to left hand while he was taking the escalator out of the subway. Once he is at home and he cannot find the stuff he bought, it's definitely somewhere between the escalator and his home. If he couldn't remember anything or didn't search his item sequentially, he may need to search his item all the way to the shop. This method is the way to reduce the scope of the area you need to search for your lost items and it'll also increase the confidence to find the lost item in the rest of search area. Again, it also has a question how one can memorize a lot of events. The answer is to have consciousness as much as possible. But how can one has that much consciousness if he has several things to concerns? That's the point and it'll lead to the first philosophy with a bigger scope. By changing 'stuffs' in the first philosophy to 'what in your head', you'll have more consciousness. Obviously, a guy without anything to pay attention can take care his valuable better than a guy who focus to memory his works or only think if his girlfriend might not love him.
\end{itemize}

\emph{There might be a few more but I'll stop here because it's one and a half page already.}
